\documentclass[usenames,dvipsnames,11pt,aspectratio=169]{beamer}
\usepackage{ifthen}
\usepackage{xcolor}
\usepackage{pgfplots}
\usepackage{amsmath}
\usepackage{centernot}
\usepackage{pifont}
\usepackage{tabularx}
\usepackage{makecell}
\usepackage{cuted}
\usepackage{subcaption}
\usepackage{booktabs}
\usepackage{array}
\usepackage{textcomp}
\usepackage{setspace}
\usepackage{xspace}
\usepackage{tikz}
\usepackage{pdfcomment}
\newcommand{\pdfnote}[1]{\marginnote{\pdfcomment[icon=note]{#1}}}
%\newcommand{\pdfnote}[1]{}

\usepackage{pgfpages}
%\setbeameroption{show notes on second screen}


\input ../beamer-style
\input ../std-macros
\input ../macros

\AtBeginSection[]
{
    \begin{frame}
        \frametitle{Table of Contents}
        \tableofcontents[currentsection]
    \end{frame}
}
\parskip=10pt

\title[CSCI-GA.2590]{Report Writing and Presentation}
\author[He He]{He He
}
\institute[NYU]{New York University}
\date{April 18, 2023}

\begin{document}
\begin{frame}
\titlepage
\end{frame}

\begin{frame}
    {Logistics}

    Project presentation (10\%): in class on May 2\\
    \begin{itemize}
        \item Format: 3 minute talk and 1 minute Q\&A with the audience
        \item Recording: slides should be shared through Zoom
        \item Grading (10\%):
            \begin{itemize}
                \item Clarity (problem 2\% + approach 3\% + evaluation 2\%)
                \item Style (clean an legible slides) (2\%)
                \item Q\&A (1\%)
            \end{itemize}
    \end{itemize}
\end{frame}

\begin{frame}
    {Logistics}
    Report (30\%): due on May 5\\
    \begin{itemize}
        \item Each group only needs to submit one report on Gradescope
        \item Format: 
            \begin{itemize}
                \item Page limit: 4
                \item File format: .pdf
                \item Style: ACL format (see templates on website)
            \end{itemize}
        \item Grading (30\%):
            \begin{itemize}
                \item Format (2\%)
                \item Clarity (6\%)
                \item Literature survey (4\%)
                \item Technical content (10\%) 
                \item Evaluation (8\%)
            \end{itemize}
    \end{itemize}
\end{frame}

\begin{frame}
    {Write the paper first}
    Jason Eisner's advice: \url{http://www.cs.jhu.edu/~jason/advice/write-the-paper-first.html}

    Writing is not just translating thinking into words; it is part of the thinking.
    \begin{itemize}
        \item Writing is the best use of limited time
            \begin{itemize}
                \item The audience only pays attention as far as they can understand
                \item The idea (the writeup) is often more useful than the software
            \end{itemize}
        \pause
        \item Writing helps you plan what to do next 
            \begin{itemize}
                \item Writing things down forces you to come up with a coherent story 
                \item It helps you see the holes in your logic (if you are honest)  
            \end{itemize}
        \pause
        \item Writing makes collaboration easier
            \begin{itemize}
                \item Meetings are more effective
                \item Someone can continue to work from where you left 
            \end{itemize}
    \end{itemize}
\end{frame}

\begin{frame}
    {Typical structure of an NLP paper/talk}
    \begin{enumerate}
        \item \textbf{Title}: a precise summary of the paper's contribution
        \item \textbf{Abstract}: high-level overview of the paper
        \item \textbf{Introduction}: extended abstract
        \item \textbf{Method}: 
            \begin{enumerate}
                \item Problem statement: formulation of the problem
                \item Approach: technical details of your method
            \end{enumerate}
        \item \textbf{Experiments}:
            \begin{enumerate}
                \item Setup: data, baselines, hyperparameters etc.
                \item Results: report all results (positive or negative)
                \item Analysis: how should one read the results 
            \end{enumerate}
        \item \textbf{Related work} (may also come after intro): prior approaches to the problem
        \item \textbf{Conclusion}: summarize the highlights and point out future directions
    \end{enumerate}
\end{frame}

\begin{frame}
    {Title}
    \begin{itemize}
        \item Used as an index to retrieve or refer to the paper
        \item Be precise and specific
            \begin{itemize}
                \item Too generic: ``A New Model for Image Caption Generation''
                \item Better: ``Neural Image Caption Generation with Visual Attention''
                \item Think about the keywords your audience might put in the search bar
            \end{itemize}
        \item Don't overclaim your contribution
        \item It's nice to have a catchy phrase but don't force it 
            \begin{itemize}
                \item Include it only if it's relevant and delivers the key message
                \item Good examples: \\
                    ``\emph{Show, Attend and Tell}: Neural Image Caption Generation with Visual Attention''\\
                    ``\emph{Know What You Don't Know}: Unanswerable Questions for SQuAD''
                \item Often a precise, ``plain'' title is good enough
            \end{itemize}
        \item Avoid extra formatting 
    \end{itemize}
\end{frame}

\begin{frame}
    {Abstract}
        Include all key messages: abstract has the most number of readers! \\
            \begin{enumerate}
                \item \textbf{The problem}:
                    \begin{enumerate}
                        \item The general topic: {\footnotesize``Deep Neural Networks (DNNs) are powerful models that have achieved excellent performance on difficult learning tasks.''}\\
                        \item The specific problem: {\footnotesize ``Although DNNs work well whenever large labeled training sets are available, they cannot be used to map sequences to sequences.''}
                    \end{enumerate}
                \item \textbf{Your contribution}: \\
                    \begin{enumerate}
                        \item One-sentence summary: {\footnotesize``In this paper, we present a general end-to-end approach to sequence learning that makes minimal assumptions on the sequence structure.''}\\
                        \item Your method: {\footnotesize ``Our method uses a multilayered Long Short-Term Memory (LSTM) to map the input sequence to a vector...''}\\
                        \item Impact/results: {\footnotesize ``Our main result is that on an English to French translation task from the WMT-14 dataset...''}
                    \end{enumerate}
            \end{enumerate}
\end{frame}

\begin{frame}
    {Introduction}
    Follow the same structure as the abstract but with more details\\
    \begin{enumerate}
        \item {\bf The problem}: context, motivation, important related work 
        \item {\bf Your approach}: \blue{intuition}, high-level description
        \item {\bf Results}: summary of experimental results, \blue{takeaways}
    \end{enumerate}
    \begin{itemize}
        \item Don't spend too much space on general introduction and related work. The reader should know what you did by the second paragraph.
        \item Use \blue{examples and figures} to illustrate your method.
        \item All claims should be supported by experiments/theories/analysis/citations.
        \item Use \blue{forward reference}:\\
            ``Our algorithm has linear time complexity at inference time \blue{(Section 2.1)}.''
    \end{itemize}
\end{frame}

\begin{frame}
    {Approach}
    \begin{itemize}
        \item Many different ways to structure the section depending on the content 
        \item Top down
            \begin{itemize}
                \item A documentation of your final method: ``Our model consists of three components: A, B, C.''.
                \item Easier to write but hard to read: where does each component come from?
            \end{itemize}
        \item Bottom up
            \begin{itemize}
                \item \blue{Build up the method} as if the readers are developing the method with you\\
                    ``A really simple way to solve the problme is to use A.''\\
                    ``However, it doesn't consider X, so we add B.''\\
                    ``To further improve Y, we add C.''
                \item By the time the reader finishes reading, they should think: ``This is obvious! I could have come up with the idea''.
            \end{itemize}
        \item Often a mix of the two
    \end{itemize}
\end{frame}

\begin{frame}
    {Approach}
    General suggestions:\\
    \begin{itemize}
        \item Always give \blue{high-level ideas and intuitions} before go into technical details
        \item Put the ``\blue{why}'' before the ``how''
        \item Use running \blue{examples} when illustrating complex concepts or procedures
        \item Notations and terminologies should be \blue{consistent} throughout the paper
        \item Use equations only when it adds additional information.
            \begin{itemize}
                \item Sometimes an idea can be described precisely in words.
                \item Even when you do need an equation, decribing it in words first is helpful.
            \end{itemize}
    \end{itemize}
\end{frame}

\begin{frame}
    {Experiments}
    \begin{itemize}
        \item {\bf Setup}: datasets, metrics, baselines
        \item {\bf Implementation details}: preprocessing, hyperparameters of the model and the algorithm
        \item {\bf Results}:
            \begin{itemize}
                \item Full results are often shown in tables and charts.
                \item In the main text, highlight important numbers and \blue{takeaways}.
            \end{itemize}
        \item {\bf Analysis}: provides better understanding of the results, e.g.
            \begin{itemize}
                \item Ablation study of the model: are all components equally important?
                \item Error analysis: what are the limitations of the method?
                \item If the results are negative: what are possible reasons?\\
                    (Often there is a mismatch between the assumptions you made and the data.)
            \end{itemize}
    \end{itemize}
\end{frame}

\begin{frame}
    {Related work}
    If some work directly motivates your work, it should go into the intro. E.g. you are\\
    \begin{itemize}
        \item extending a previous work,
        \item applying a prior method to your problem.
    \end{itemize}

    Why put related work at the end?\\
    \begin{itemize}
        \item The reader has not gained a good understanding of the problem and your approach to make judgments after the introduction.
        \item It breaks the flow from the introduction to the approach.
        \item It provides an overview of the area and transitions well to the conclusion.
    \end{itemize}

    Don't just survey the area; describe \blue{how your work situates in the broader context}.
\end{frame}

\begin{frame}
    {Conclusion}
    \begin{itemize}
        \item Highlight the key findings from your work.
        \item Here is the place where you can take some freedom to state your opinions
            \begin{itemize}
                \item What are the limitations of your approach?
                \item What are the important next steps?
                \item What are promising future directions for the problem?
            \end{itemize}
    \end{itemize}
\end{frame}

\begin{frame}
    {General suggestion}
    \begin{itemize}
        \item Think about the key message you want to deliver and ``repeating'' it throughout the paper.
        \item Put yourself in the reader's shoes.
        \item Get feedback from people who are less familiar with your work.
    \end{itemize}
\end{frame}

\begin{frame}
    {Talks vs papers}
    \begin{itemize}
        \item The talk mirrors the paper so much of the advice still applies.
        \item But, the talk contains \emph{much fewer details}.\\
            The paper is similar to the manual of a tool,
            while the talk is your explanation of how to use the tool.
        \item You have more freedom on how to communicate in the talk
            \begin{itemize}
                \item Use \blue{visual aids}: video/audio/animation, memes/stories etc.
                \item \blue{Interact} with the audience (ask questions, voting, demo)
            \end{itemize}
        \item Purpose of the talk
        \begin{itemize}
            \item Convey the key idea of your paper (so that they will go read it)
            \item Entertaining (it's a performance)
        \end{itemize}
    \end{itemize}
\end{frame}

\begin{frame}
    {Giving the talk}
    General advice\\
    \begin{itemize}
        \item Number one mistake to avoid: \textcolor{red}{go over the time limit}\\
            (Generally, prepare $n$ slides for an $n$-minute talk.)
        \item Know your opening sentences by heart (the rest will be easy)
        \item Use a lot of \blue{examples}, especially given limited time
        \item Be brief on related work (they may be mentioned during Q\&A)
    \end{itemize}
    For the project presentation\\
    \begin{itemize}
        \item You only have 3 minutes: practice is important!  
        \item If more than one member will be presenting, practice the transition.
        \item Test audio and screen sharing to avoid technical issues.
        \item Suggested structure:\\
            problem/motivation, key ideas, evaluation, conclusion (1 slide each)
    \end{itemize}
\end{frame}

\begin{frame}
    {Opening}
    Main goal: engage the audience\\
    \begin{itemize}
        \item What is the problem? (Audience: what's this about?)
        \item Why is it important? (Audience: why should I care?)
        \item What's the challenge? (Audience: this is not trivial?)
    \end{itemize}
    They should be excited to hear your approach by now.

    Be creative: use examples, data, quotes, anecdotes etc.
\end{frame}

\begin{frame}
    {Key ideas}
    \begin{itemize}
        \item \emph{Use the bottom-up strategy}: solve the puzzle with the audience. \\
                It doesn't have to follow how you actually solved the problem,
                but there should be a story to connect all the pieces.
            \item \emph{Think about what the audience may be wondering}: ask a question, then answer it. \\
            ``Now, how do we optimize it given that the expectation is intractable? We use ...''
        \item \emph{Leave the details to the paper}: what is the minimal set of things they need to understand?
        \item \emph{Create references} to your key ideas/concepts.
            \begin{itemize}
                \item Use icons, figures, running examples, toy problems, analogies etc.
                \item This should be in their head when they retrieve your idea.
            \end{itemize}
    \end{itemize}
\end{frame}

\begin{frame}
    {Results}
    \begin{itemize}
        \item Don't paste a wall of numbers. Use bar charts.
        \item Guide the audience on where they should pay attention.
        \item Put an explicit takeaway/tagline for each result you show.
        \item Think about it as a sequence of question and answers.\\
            ``How well does the approach work on benchmark A?''\\
            ``What about a different domain?''\\
            ``Is component B really necessary?''\\
            ``How fast is the inference?''
    \end{itemize}
\end{frame}

\begin{frame}
    {Conclusion}
    \begin{itemize}
        \item Acknowledgement to collaborators.
        \item Show the takeaway messages you want the audience to leave with.
        \item Show the highlights/summary of the talk to start Q\&A. 
    \end{itemize}
\end{frame}

\begin{frame}
    {Style}
    Basics\\
    \begin{itemize}
        \item Font: sans-serif; sizes: 24+.
        \item Number slides (for easy reference during Q\&A).
    \end{itemize}

    Layout\\
    \begin{itemize}
        \item Avoid clutter, wordiness (and full sentences).\\
              The lecture slides are bad examples (because it's used both for the talk and for reference).
        \item Use keywords, diagrams, figures: the audience should pay attention to \textit{you}, not the slides.
        \item If the content is complex, use animation to reveal the full slide.
    \end{itemize}

    Visual guidance\\ 
    \begin{itemize}
        \item Color coding should be consistent throughout the presentation.
        \item Use arrows and boxes to help the audience read figures and tables.
        \item Highlight (bold, background color) key messages.
    \end{itemize}
\end{frame}

\begin{frame}
    {Q\&A}
    \begin{itemize}
        \item Common types of questions
            \begin{itemize}
                \item Clarification: ``Can you explain X again?''
                \item Interpration of the results: ``Why is there a dip in the learning curve?''
                \item Comparison: ``How is this different from X?''
                \item Extensions and new perspectives of the work (this is the type of questions we like)
            \end{itemize}
        \item It's okay to pause before you answer the question.
        \item Provide your thoughts even if you have no idea what's the answer.
    \end{itemize}
\end{frame}

\end{document}
